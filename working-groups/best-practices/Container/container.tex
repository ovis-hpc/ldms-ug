\documentclass{article}
\usepackage{graphicx} % Required for inserting images
\usepackage{hyperref}
\title{ldms-ug-best-container}
\author{Thomas Tucker}
\date{October 2023}

\begin{document}
\section{Containers}
\subsection{Build Containers}
These containers include all of the tools and dependencies necessary to build the top of tree on a white-box Linux platform. These will work on Cray platforms, but for reasons of proprietary library licensing (e.g. Slingshot), they may not be able to build transport plugins for proprietary networks.

The build containers are useful for building other containers as described below
\subsection{Generic Containers (“Runtime” containers?)}
These containers provide a platform for deploying LDMS Aggregators and Samplers but with your own configuration. They contain pre-built Maestro and ldmsd along with transport, sampler, authentication and storage plugins.

\subsubsection{Transport Plugins}
\begin{itemize}
    \item socket
\item infiniband
\item libfabric
\end{itemize}
\subsubsection{Authentication Plugins}
\begin{itemize}
    \item none
    \item ovis (we need to rename this or create some reasonable alias such as ‘shared-secret’)
    \item munge
\end{itemize}
\subsubsection{Sampler Plugins}
\begin{itemize}
\item meminfo
\item vmstat
\item procstatutil
\item …
\end{itemize}

\subsection{Storage Plugins}
\begin{itemize}
\item SOS
\item CSV
\item Avro\_Kafka
\item . . .
\end{itemize}

\section{Demo/Pre-configured Containers}
These containers are derived from the Generic Containers.

These are pre-configured deployments of a Sampler, and an Aggregator. The transport configuration will be sockets. The intention is that a user could trivially stand up a monitoring infrastructure with very little effort. These containers are also a useful starting point for creating a customized deployment, for example over Infiniband instead of sockets.

Reasonable defaults are included for all tunables, for example sampler interval. These can be overridden when the container is started with arguments to the docker run command. For example:

\begin{verbatim}
    docker run sampler i_am=001 interval=10s
\end{verbatim}

Host naming will be normalized and controlled by parameters passed to the docker run command, e.g.:

\begin{verbatim}
    docker run aggregator \
        i_am=agg-01 \
        samplers=sampler-[001-100] \
        storage=/shared/agg-01/ldmsd
\end{verbatim}

\subsection{Status}
The following containers are currently available and maintained on docker-hub:
\begin{itemize}
\item https://hub.docker.com/r/ovishpc/ldms-dev
\item https://hub.docker.com/r/ovishpc/ldms-samp
\item https://hub.docker.com/r/ovishpc/ldms-agg
\item https://hub.docker.com/r/ovishpc/ldms-storage
\item https://hub.docker.com/r/ovishpc/ldms-web-svc
\item https://hub.docker.com/r/ovishpc/ldms-grafana
\item https://hub.docker.com/r/ovishpc/ldmscon2023
\end{itemize}
All of the containers are based on ubuntu:22.04.

The github repository for the source code for building the containers can be found at https://github.com/ovis-hpc/ldms-containers

\end{document}
