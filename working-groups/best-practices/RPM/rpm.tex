\documentclass{article}
\usepackage{graphicx} % Required for inserting images
\usepackage{hyperref}
\title{ldms-ug-best-rpm}
\author{Thomas Tucker}
\date{October 2023}

\begin{document}
    \section{Assumptions}
        \begin{itemize}
                \item Python3 and python3-devel are required to build all RPM
                \begin{itemize}
                    \item python3 support is presumed to be present on all target systems
                \end{itemize}
                \item Any of the ‘standard’ packages that don’t build will break the RPM build process
        \end{itemize}
   \section{RPM Version Consistency}
        \begin{itemize}
                \item ldms\_ls -V
                \item ldmsd -V
                \item Should be able to identify the RPM and/or local build that is currently running
                \item configure must be run to update the CPP macros that are used by software to report version if you don’t re-run configure, you’ll get “old” information
                \item should implement procedure to appropriately create tags where necessary
        \end{itemize}        
    \section{Naming Conventions}
        \begin{itemize}    
            \item There are lots of docs on this sort of thing that we should comply with
                \begin{itemize}
                    \item \href{http://ftp.rpm.org/max-rpm/ch-rpm-file-format.html}{http://ftp.rpm.org/max-rpm/ch-rpm-file-format.html}
                    \item \href{https://docs.fedoraproject.org/en-US/packaging-guidelines/Versioning/}{https://docs.fedoraproject.org/en-US/packaging-guidelines/Versioning/}
            \end{itemize}
            \item If git hash matches release level:
                \begin{itemize}
                    \item Major.minor.fix
                \end{itemize}
            \item otherwise
                \begin{itemize}
                        \item major.minor.fix-<patchcount>-<short-hash>
                        \item Rpm packagers often have to create a local tag with somehow meaningful version name (if they want to avoid seeing patch-hash suffix.
                        \item Currently the short-hash is not computed correctly in all cases.
                \end{itemize}
            \item make dist tar file output naming must match ldms\_ls -V output
        \end{itemize}
    \section{Platform/Software Independent RPM Packages}
        \begin{itemize}
            \item Commands, tools and dependent libraries
            \item Any sampler that will build regardless of dependencies
                \begin{itemize}
                    \item meminfo
                    \item vmstat
                    \item procstat
                    \item procnetdev
                    \item ??
                \end{itemize}
            \item Any store sampler that will build without dependencies
               \begin{itemize}
                    \item store
                    \item json
                    \item flatfile
                    \item store\_none
                    \item store\_function
                    \item ??
                \end{itemize}
            \item RPM for transports that are “always” available
                \begin{itemize}
                    \item socket
                \end{itemize}
        \end{itemize}
    \section{RPM Packages Hardware/Platform Specific Samplers}
        \begin{itemize}
            \item IB package
            \item Lustre package
            \item Slingshot package
            \item slurm package
            \item Aries package
            \item Cray package
            \item Nvidia package
        \end{itemize}
    \section{RPM Packages For Stores}
        Many of these storage plugins will/should be deprecated as part of moving the core technology stores to support decomposition
        \begin{itemize}
            \item avro\_kafka
            \item darshan
            \item influx
            \item json
            \item kafka
            \item kokkos
            \item kokkos\_appmon
            \item papi
            \item proc\_store
            \item rabbit
            \item slurm
            \item store\_amqp.c
            \item store\_app
            \item store\_rabbitkw.c
            \item store\_rabbitv3.c
            \item store\_sos.c
            \item stream
            \item stream\_dump
        \end{itemize}
    \section{SNL package split}
        \begin{itemize}
            \item main(all binaries that got built using/requiring the ldmsd binary)
            \item Python layer (maybe subsumed in main)
            \item Doc layer (man, html, etc)
            \item Any local value-added plugin rpm(s).
            \item Daemon configuration rpm (genders or maestro or ansible or whatever)
            \item All the above could be controlled by rpm ‘with’ options and merged into a single rpm output at the whim of the system owner who builds their own deviant rpms.
        \end{itemize}
\end{document}
